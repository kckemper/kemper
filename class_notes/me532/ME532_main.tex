\documentclass[11pt]{article}
\setlength{\textwidth}{4.5in}
\setlength{\textheight}{9in} %  
\setlength{\hoffset}{-.5in} %  
\setlength{\topmargin}{-.5in}

\usepackage{graphics,graphicx}
\usepackage{amsmath} 
\newcommand{\dotp}{\,\mathbf{\cdot}\,}
\newcommand{\cross}{\times}
\newcommand{\grad}{\nabla}
\newcommand{\diver}{\grad \dotp}
\newcommand{\pd}{\partial}
\newcommand{\ds}{\displaystyle}
\pagestyle{myheadings}
\markright{Optimal \& Robust Control, B. Batten}

\begin{document}
%\title{Notes for ME 532 Optimal and Robust Control}
%\author{Belinda A Batten\\ School of Mechanical, Industrial, \& Manufacturing Engineering\\
%Oregon State University\\
%Corvallis, Oregon 97331-6011\\
%Email: bbatten@engr.orst.edu}

%\maketitle
%\begin{abstract}  This course is the second in the ME graduate control sequence on linear methods. The course focuses on developing methods from optimization to derive optimal %controls. Those approaches will be used to develop robust controls such as MinMax and $H_{\infty}$. Some numerical methods will be demonstrated via Matlab, and each student %will be expected to have a project to which they can apply methods discussed in the course.
%\end{abstract}

%\section{Introduction}
%\label{sec:introduction}
%General unconstrained optimization problems from calculus typically take the form of
\begin{equation}
\min f(x)\, {\rm or}\, \max f(x).
\end{equation}
Constraints can be added and the problems take the form of 
$$
\min f(x)\, {\rm or}\, \max f(x).
$$ 
subject to
\begin{equation}
g(x) = C.
\end{equation}
ME 517 considers cases where these problems are algebraic (i.e., no differential equations). The functions $f, g$ can be defined on scalars $x\in \Re$, or vectors $x \in \Re^n$; similar, $f, g$ can take on values in $\Re$ or in $\Re^n$.

In this course, there may be derivatives in either the objective function $f$, or in the constraints $g$. We will develop techniques to address these kinds of problems, and we will need the basics of the Calculus of Variations. The types of problems addressed in this course will be those for which a mathematical model can be developed. 

Examples of optimal control problems:
\begin{enumerate}
\item About 1910, R. H. Goddard asked if we can do space research with rockets. Was involved in the liquid propellant rocket engine whose thrust could be controlled. Question: How should thrust vary with time in order to reach the highest altitude?  \\
Objective function:  maximize altitude at final time, $h(t_f)$\\
Constraint:  with no drag, the thrust should be maximum until no propellant is left.  However, if drag depends on velocity and height, maybe high velocity is not good in densest part of atmosphere. Optimal control can be used to develop a velocity profile.
$v$: velocity, $u$: thrust, $h$: height, $m$: mass, $g$: gravity, $D$: drag
$$\max h(t_f)$$
\hspace{1.5in}{\it subject to}
\begin{eqnarray*}
\dot{v}&=&\frac 1m(u - D(v,h) - g)\\
\dot{m}&=&-\gamma u
\end{eqnarray*}
$$
v(0) = 0, \quad h(0) = 0, \quad m(0) = m_0, \quad m(t_f) = m_1$$
$$0\leq u\leq u_{max}$$

\item Elevated train $m$ on a smooth track with friction constant $\gamma$. Bring train to rest in minimum time with minimum jerk.\\
$x(t)$: position of train; $u(t)$: external force controlling train.
$$\min \sigma t_f + (1-\sigma)\dddot x(t)$$
\hspace{1.5in}{\it subject to}
$$ m\ddot x (t) + \gamma \dot x(t) = u(t)$$
$$x(0) = x_0, \quad \dot x (0) = x_1, \quad x(t_f) = x_d, \quad \dot x(t_f) = 0, \quad  |u(t)| \leq u_{max}$$

\item Linear Quadratic Regulator---infinite time horizon version
$$\min \int_0^{\infty} \left(x^T(t)Qx(t) + u^T(t)Ru(t) \right)dt$$
\hspace{1.5in}{\it subject to}
$$\dot x(t) = Ax(t) + Bu(t), \quad x(0) = x_0$$
$$ x(t) \in \Re^n, \quad u(t) \in \Re^m, \quad A\in\Re ^{n\times n}, \quad B\in \Re ^{n\times m}$$

Example:  $m\ddot y + d\dot y + k y = u(t), \quad y(0) = y_0, \dot y(0) = y_1$
\begin{eqnarray*}
 x_1 = y \qquad \dot x_1 = \dot y &=& x_2 \\
 x_2 =\dot y \qquad \dot x_2 = \ddot y &=& -\frac d m \dot y - \frac km y + \frac 1m u\\
 &=& -\frac d m x_2 - \frac km x_1 + \frac 1m u
\end{eqnarray*}

\begin{eqnarray*}
\dot x = \left[\begin{array}{c} \dot x_1\\ \dot x_2\end{array}\right] = \left[\begin{array}{cc}0  & 1\\  - \frac km &-\frac d m\end{array}\right] \left [ \begin{array}{c} x_1\\ x_2\end{array}\right]  + \left[\begin{array}c 0 \\ \frac 1m\end{array}\right] u
\end{eqnarray*}
 If $\ds Q = \left[\begin{array}{cc} 0&0\\0&1\end{array}\right]$, then velocity is weighted;  if $\ds Q = \left[\begin{array}{cc} 1&0\\0&0\end{array}\right]$, then position is weighted.
 
 



\end{enumerate}

%\section{Linear Quadratic Regulator, take I}
%\label{sec:lqr}
%The Linear Quadratic Regulator (LQR) Problem can be written as
$$\min J(x,u) = \int_0^{T} \left(x^T(t)Qx(t) + u^T(t)Ru(t) \right)dt$$
\hspace{1.5in}{\it subject to}
$$\dot x(t) = Ax(t) + Bu(t), \quad x(0) = x_0$$
$$ x(t) \in \Re^n, \quad u(t) \in \Re^m, \quad A\in\Re ^{n\times n}, \quad B\in \Re ^{n\times m}$$
$$Q\in\Re ^{n\times n}, \quad R\in \Re ^{m\times m}\quad T \hspace{.1in}{\rm fixed}.$$
The matrix $Q$ must be positive semidefinite, and the matrix $R$ must be positive definite for the problem to have a solution.

Since $x$ is a function of $u$, we can consider the objective function alternatively as a function of $u$, i.e., $J(u)$. Using tools which we will develop later in the
course, we can write the solution for this problem in feedback form as 
$$u = Kx$$
where $K = -R^{-1}B^T \Pi$ and $\Pi$ is the solution of the differential Riccati equation
\begin{equation}\label{eq:dre}
\dot{\Pi}(t) +A^T\Pi(t) +\Pi(t)A - \Pi(t)BR^{-1}B^T\Pi(t) +Q =0, \hspace{.1in}  \Pi(T) = 0.
\end{equation}
If we let $T\rightarrow\infty$, the steady state form of the problem is the same, but $\Pi$ is then the solution of the algebraic Riccati equation
\begin{equation}\label{eq:are}
A^T\Pi(t) + \Pi(t)A - \Pi(t)BR^{-1}B^T\Pi(t) +Q=0.
\end{equation}
Note that if $Q$ is positive definite, $\Pi$ is also positive definite.


%
\setcounter{page}{7}
\setcounter{section}{4}
%\section{Optimization Review}
%\label{sec:opt}
%\subsection{Unconstrained Optimization Problems}
\begin{itemize}
\item Functions of a single variable. Let the function $f: \Re \rightarrow \Re$. First order necessary conditions for a minimum at $x = x^*$
$$f'(x) = \frac{d}{dx}f(x^*) = 0.$$
Second order sufficiency conditions
$$f''(x) = \frac{d^2}{dx^2}f(x^*) > 0.$$
If $f:[a,b]\rightarrow\Re$, then the minimum could occur at $a$ or $b$.


\item Functions of multiple variables. Let the function $f: \Re^n \rightarrow \Re$, example $f(x_1, x_2, x_3):\Re^3 \rightarrow \Re$. First order necessary conditions for a minimum at $x = x^*$
$$\grad f(x^*) = 0.$$ Second order sufficiency conditions  $$\grad^2 f(x^*) >0, $$ that is, the matrix is positive definite. To be precise,
$$\grad f(x^*) = \left[ \frac{\pd f}{\pd x_1}(x^*), \frac{\pd f}{\pd x_2}(x^*), ..., \frac{\pd f}{\pd x_n}(x^*)\right]^T = 0$$
$$\grad^2 f(x^*) = \left[\begin{array}{cccc} \frac{\pd^2 f}{\pd x_1^2}(x^*)& \frac{\pd f^2}{\pd x_1 \pd x_2}(x^*)& ... & \frac{\pd f^2}{\pd x_1\pd x_n}(x^*)\\[1em]
\frac{\pd^2 f}{\pd x_2\pd x_1}(x^*)& \frac{\pd f^2}{\pd x_2^2}(x^*)& ... & \frac{\pd f^2}{\pd x_2\pd x_n}(x^*)\\[2em]
\frac{\pd^2 f}{\pd x_n\pd x_1}(x^*)& \frac{\pd f^2}{\pd x_n \pd x_2}(x^*)& ... & \frac{\pd f^2}{\pd x_n^2}(x^*)
\end{array}\right] {\rm positive\, definite}.$$


\end{itemize}
Example problems
\newpage

\subsection{Optimization with Static Equality Constraints}
Standard problem takes the form
$$\min f(x), f:\Re^n \rightarrow \Re$$
\hspace{.15in}{\it subject to}
$$ g(x) = C, g:\Re^n\rightarrow\Re^m$$
Example:  $$\ds \min f(x) = \frac 12 \left(\frac{x_1^2}{a^2}+\frac{x_2^2}{b^2}\right)$$
\hspace{1.5in}{\it subject to} $$x_1 + mx_2 = C$$

\begin{itemize}
\item Solution approach 1: eliminate constraint through substitution and treat as unconstrained problem
\newpage
\item Solution approach 2: look at geometry of problem.  A minimizer can only occur where constrain set is tangent to $f$. At these tangent points, the gradient of $f$ is parallel to the gradient of $g$. Hence, there exists a constant $\lambda$ so that $\grad f(x^*) = \lambda \grad g(x^*)$ (if $g$ maps to $\Re^m$, then $\lambda$ is a vector of size $m$). Define the Lagrangian to be $$H(x) = f(x) + \lambda^Tg(x), \quad {\rm for} \lambda \in \Re^m.$$  The gradient of $H$ is zero at $x^*$ means that the gradient of $f$ is parallel to the gradient of $g$ at $x^*$. 
\end{itemize}
Example continued...
\newpage

%
\section{Optimal Control Problems}
\label{sec:optcont}
\subsection{Basic  Problem in Optimal Control:}
$$\min J(x,u)=\int_{t_0}^{t_f} L(t,x(t),u(t)) dt + \varphi(x(t_f),t_f)$$
\hspace{1.5in}{\it subject to}
$$\dot x(t) = f(t,x(t),u(t)), \qquad  x(t_0) = x_0.$$

To work up to solving this, we'll follow the pattern of what we did with algebraic optimization problems.  Instead of having objective functions that map from $\Re^n$ into $\Re$, the objective functions can be functionals.  Functionals map functions to real numbers. The calculus required to find minima for functionals is called the calculus of variations (COV).

\noindent {\bf Fundamental Problem in the Calculus of Variations: } Given a functional $J$ and a set of admissible functions $A$, determine which function (or functions) provide a minimum value for $J$.

\bigskip
\noindent Example problems:
\begin{enumerate}
\item Find the function $f$ with minimum arclength in the set $A$ of all continuously differentiable functions on $a\leq x\leq b$ with $f(a) = f_1$, $f(b) = f_2$. Define
$$J(f) = \int_a^b \sqrt{1+f'(x)^2}dx$$
Then, we want to find the $f$ that minimizes $J$.
\item A bead with mass $m$ and initial velocity $v(0) = 0$ slides with no friction under the force of gravity from point $(x_1, y_1)$ to $(x_2, y_2)$ along a wire defined by a curve $f(x)$. What is the shape that has the shortest travel time?
$$T = \int_0^Tdt = \int_0^{s_1} \frac{dt}{ds} ds = \int_0^{s_1} \frac 1v ds = \int_{x_1}^{x_2}\frac{ \sqrt{1+f'(x)^2}}vdx.$$
To determine $v$ in terms of $f$ or $x$, use the fact that energy is conserved. Energy in the system:
$$\frac 12 mv^2 + mgf = 0 + mgy_1$$
at the starting point; this implies that
$$v = \sqrt{2g(y_1-f(x))}.$$
So for any curve, the time to traverse the curve is given by
$$T(f) = \int_{x_1}^{x_2}\frac{ \sqrt{1+f'(x)^2}}{\sqrt{2g(y_1-f(x))}}dx$$
\item Linear Quadratic Regulator with sensed measurements
$$\min \int_0^{\infty} \left(x^T(t)Qx(t) + u^T(t)Ru(t) \right)dt$$
\hspace{1.5in}{\it subject to}
$$\dot x(t) = Ax(t) + Bu(t), \quad x(0) = x_0$$
$$ y(t) = Cx(t)$$
$$ x(t) \in \Re^n, \quad u(t) \in \Re^m, \quad A\in\Re ^{n\times n }, \quad B\in \Re ^{n\times m}, \quad C\in\Re^{n\times p}$$
$Q$ positive semidefinite, $R$ positive definite.
\end{enumerate}

To minimize these problems, we need to develop necessary and sufficient conditions. There are many intricacies to this, and we will just scratch the surface. 

{\it Key point to remember:} Analogously to the calculus used for basic optimization problems, we would like to have a condition like ``$J'(f)=0$" as a way to find candidate minimizers $f^*$.  We will return to this idea repeatedly to develop the ideas that can be used to solve optimal control problems.

\medskip

Suppose $f^*$ minimizes $J$ locally. Then we can generate other admissible functions ``nearby $f^*$" in the set $A$ by $f = f^* + \epsilon h$ where $\epsilon$ is a small real number and $h$ is a function. If we pick $h$ carefully and keep $\epsilon$ small, $f$ will remain in $A$.

 If $J$ is minimized at $f^*$ then ${\cal J}(\epsilon)=J(f^* +\epsilon h)$ is minimized when $\epsilon = 0$. So, related to $J(f)$, we have created a related function $\cal J(\epsilon)$. We can apply standard calculus to minimizing $\cal J$ and thus uncover necessary conditions for $J(f)$. In particular, we know that ${\cal J}(\epsilon)$ is minimized when $\epsilon = 0$.  Therefore the first order necessary condition for a minimum for $\cal J$ is ${\cal J}'(0) = 0$.  The function $f^*$ is called the {\it extremizer} (or candidate minimizer) for $J$ and $J$ is called {\it stationary} at $f^*$.
 
\subsection{\bf Simplest problem in the calculus of variations:}
 Find a local minimum for $$J(f) = \int_a^b L(x,f,f')dx$$ where $f\in C^2[a,b]$, $f(a) = y_1$, $f(b) = y_2$, $L$ is a given function that is $C^2$ on $[a,b]\times \Re^2$. Note that $f$ is smoother than needed if a more general approach is used. It is easier to develop the necessary conditions with this assumptions, but not applicable to as wide a variety of problems.
 
 Form ${\cal J}(\epsilon) = J(f^* + \epsilon h)$. We know that $\cal J$ is optimized when $\epsilon=0$ and so the first order necessary condition is ${\cal J}'(0) = 0$. 
 $${\cal J}(\epsilon) = J(f^* + \epsilon h) = \int_a^b L(x, (f^*+\epsilon h), ( f^*+\epsilon h)') dx.$$ 
 
 First order necessary condition:  $\ds \frac{d}{d\epsilon}{\cal J}(0) = 0$ corresponds to the candidate minimizer $f^*$, and $J$ is stationary at $f^*$.   Assume that $h$ is chosen to be twice continuously differentiable with $h(a) = 0 = h(b)$. Then for $\epsilon$ small enough, $f^* + \epsilon h$ remains in $A$ and is called an {\it admissible variation}.
$$ \frac{d}{d\epsilon}{\cal J}(\epsilon)\vert_{\epsilon=0} = \frac{d}{d\epsilon}J(f^* + \epsilon h)\vert_{\epsilon=0}  = \int_a^b \frac{\pd}{\pd\epsilon}L(x, (f^*+\epsilon h), ( f^*+\epsilon h)') dx\vert_{\epsilon=0} $$ 
$$  = \int_a^b \left(\frac{\pd L}{\pd x}(x, (\cdot), (\cdot)')\frac{dx}{d\epsilon} + \frac{\pd L}{\pd f}(x, (\cdot), ( \cdot)')\frac{df}{d\epsilon} + \frac{\pd L}{\pd f'}(x, (\cdot), (\cdot)')\frac{df'}{d\epsilon}\right)dx\vert_{\epsilon=0}$$
$$  = \int_a^b \left( \frac{\pd L}{\pd f}(x, f^*, (f^*)')h + \frac{\pd L}{\pd f'}(x, f^*, (f^*)')h'\right)dx = 0$$
This last line is the first order initial condition and will hold for all $h\in C^2[a,b]$ with $h(a) = 0 = h(b)$.
To put this in a more usable form, integrate by parts on the second term to change $h'$ to $h$.
\vspace{.25in}

Set $\ds u =  \frac{\pd L}{\pd f'}$ and $dv = h'dx.$  Then $\ds du = \frac{d}{dx} \frac{\pd L}{\pd f'} dx$ and $v = h$. This implies that 
\begin{eqnarray}
 0 &=& \int_a^b \left( \frac{\pd L}{\pd f}(x, f^*, (f^*)')h + \frac{\pd L}{\pd f'}(x, f^*, (f^*)')h'\right)dx \nonumber\\
 &=& 
\int_a^b \left( \frac{\pd L}{\pd f}(x, f^*, (f^*)') - \frac{d}{dx}\frac{\pd L}{\pd f'}(x, f^*, (f^*)')\right)hdx \label{eq:keyeqn}
\\&&+\frac{\pd L}{\pd f'}(x, f^*, (f^*)')h\vert_{a}^{b} \nonumber
\end{eqnarray}

\noindent To use this, we need to apply the

\noindent {\it Fundamental Lemma of the Calculus of Variations}: If $f$ is continuous on $[a,b]$ and if $\ds \int_a^b f(x) h(x) dx = 0$ for every $h \in C^2[a,b]$ with $h(a) = 0 = h(b)$, then $f(x) = 0$ for $x\in[a,b]$.

\medskip
\noindent From this, we can conclude that $$\frac{\pd L}{\pd f}(x, f^*, (f^*)') - \frac{d}{dx}\frac{\pd L}{\pd f'}(x, f^*, (f^*)')= 0. $$

To summarize this, we can write the following theorem which represents the First Order Necessary Condition for the Simplest Optimal Control problem:

\noindent{\it Theorem}: If $f^*$ provides a local minimum to $$J(f) = \int_a^b L(x, f, f') dx$$ where $f\in C^2[a,b]$ and $f(a)=y_1$, $f(b) = y_2$ then $f$ must satisfy 
\begin{equation}\label{eq:eleqn}
\frac{\pd L}{\pd f}(x, f^*, (f^*)') - \frac{d}{dx}\frac{\pd L}{\pd f'}(x, f^*, (f^*)')= 0 \quad {\rm for} \quad x \in [a,b].
\end{equation}
Equation (\ref{eq:eleqn}) is called the {\it Euler-Lagrange Equation}. Solutions $f^*$ to the Euler-Lagrange equation are called {\it extremals} and $J$ is {\it stationary} at such functions.

\noindent Examples:

\newpage

\subsection{Variation on the simplest problem:  higher derivatives of $f$ in the objective function} If
$$J(f) = \int_a^b L(x, f(x), f'(x), ..., f^{(n)}(x)) dx$$ with boundary conditions
$$f(a) = y_1, f'(a) = y_2, ..., f^{(n-1)}(a) = y_n$$
$$f(b) = z_1, f'(b) = z_2, ..., f^{(n-1)}(b) = z_n,$$
then the 1st order necessary condition becomes
$$\frac{\pd L}{\pd f} - \frac{d}{dx}\frac{\pd L}{\pd f'} + \frac{d^2}{dx^2}\frac{\pd L}{\pd f''} - \cdots + (-1)^n\frac{d^n}{dx^n}\frac{\pd L}{\pd f^{(n)}} = 0 \quad {\rm for} \quad x \in [a,b].$$
where each of the terms in the necessary condition are evaluated at $(x, f^*, (f^*)',\ldots, (f^{*(n)})$. This equation is solved to find the {\it candidate minimizers} or {\it extremals}, $f^*$, for the objective function.
\medskip

\noindent Example
\newpage

\subsection{Variation 2 on the simplest problem:  multiple functions $f, g$ in the objective function} If
$$J(f) = \int_a^b L(x, f(x), f'(x), g(x), g'(x)) dx$$ with boundary conditions
$$f(a) = y_1, f(b) = y_2, g(a) = z_1, g(b) = z_2,$$
then the 1st order necessary condition takes the form of a system of Euler-Lagrange equations for $f$ and $g$:
$$
\frac{\pd L}{\pd f}(x, f^*, (f^*)',g^*, (g^*)')) - \frac{d}{dx}\frac{\pd L}{\pd f'}(x, f^*, (f^*)',g^*, (g^*)'))= 0$$
$$\frac{\pd L}{\pd g}(x, f^*, (f^*)',g^*, (g^*)')) - \frac{d}{dx}\frac{\pd L}{\pd g'}(x, f^*, (f^*)',g^*, (g^*)'))= 0,$$
 for $ x \in [a,b]$.
These equations (typically coupled) are solved to find the {\it candidate minimizers} or {\it extremals}, $f^*, g^*$, for the objective function.


\medskip



\noindent Example
\newpage

\subsection{Variation 3 on the simplest problem:  unspecified boundary condition} Instead of having $f(a)$ and $f(b)$ specified, assume $f(a) = y_1$ and $f(b)$ is unspecified. This means that the admissible variations, $h$, have $h(a) = 0$ and $h(b)$ is unspecified. Then in order for (\ref{eq:keyeqn}) to reduce to the form to which we can apply the FLCOV, we need to assume that 
$$\frac{\pd L}{\pd f'}(b, f^*(b), (f^*(b))') = 0$$
and the Euler-Lagrange equation will still hold. 

Example

\newpage
\subsection{Hamiltonian Theory}
One method to obtain equations of motion for a physical system is to use the idea that the system should evolve along a path of least resistance. The concept of {\it action} was proposed (where action has the units of energy $\times$ time), and the motion is to minimize the action (a.k.a. Principle of Least Action).

Define $y_1, y_2, \ldots, y_n$ to be the coordinates of a given system dynamics. Then $\dot y_1, \dot y_2, \ldots, \dot y_n$ are the corresponding velocities. Denote the kinetic and potential energies of the system as $T$ and $V$ respectively.
\smallskip

\noindent Hamilton's principle states that the time evolution of a mechanical system takes place so that the integral of the difference of kinetic and potential energy is {\it stationary}.  Or, the motion of the system from time $t_0$ to $t_1$ provides an extremal for the functional
$$J(y_1, \ldots, y_n, \dot y_1, \ldots, \dot y_n) = \int_{t_0}^{t_1} (T-V)dt$$

The functions $y_1(t), \ldots, y_n(t)$ can also be thought of as parametric equations defining the paths that the system takes over time.

\medskip

Examples



%
%
%\section{Numerical Results}
%\label{sec:results}
%\input{results}
%
%\newpage
%\pagestyle{empty}
%\section*{Notation}
%\label{sec:notation}
%

\begin{itemize}
\item $\Re$:  the set of real numbers, also written as $x\in\Re$ iff $-\infty<x<\infty$
\item $\Re^n = \Re \times \Re \times \cdot \times\Re$; a vector $x \in \Re^n$ iff each element of the vector $x_i\in \Re$.
\item $A\in \Re^{n\times m}$ denotes a matrix with $n$ rows and $m$ columns
\item $C[a,b]$ denotes the set of functions that are continuous on the interval $a\leq x\leq b$
\item $C^n[a,b]$ denotes the set of functions that has $n$ continuous derivatives on the interval $a\leq x \leq b$
\item Leibnitz's formula: $$\ds \frac{d}{dx}\int_{g(x)}^{h(x)}L(x,f(x))dx = \int_{g(x)}^{h(x)}\frac{\pd}{\pd x}L(x,f(x))dx$$ 
 $$ \qquad+ \frac{d h(x)}{d x}L(h(x),f(h(x)))-\frac{d g(x)}{d x}L(g(x),f(g(x))$$

\end{itemize}

%
%----------------------------------------------------------------

%\section*{Acknowledgments}
%This research is supported in part by the Air Force Office of Scientific Research through grants FA9550-05-1-0041 and FA9550-07-1-0540.

\bibliographystyle{aiaa}
\bibliography{AIAA09}


\end{document}


\end{document}

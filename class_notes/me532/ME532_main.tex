\documentclass[11pt]{article}
\setlength{\textwidth}{4.5in}
\setlength{\textheight}{9in} %  
\setlength{\hoffset}{-.5in} %  
\setlength{\topmargin}{-.5in}

\usepackage{graphics,graphicx}
\usepackage{amsmath} 
\newcommand{\dotp}{\,\mathbf{\cdot}\,}
\newcommand{\cross}{\times}
\newcommand{\grad}{\nabla}
\newcommand{\diver}{\grad \dotp}
\newcommand{\pd}{\partial}
\newcommand{\ds}{\displaystyle}
\pagestyle{myheadings}
\markright{Optimal \& Robust Control, B. Batten}

\begin{document}
%\title{Notes for ME 532 Optimal and Robust Control}
%\author{Belinda A Batten\\ School of Mechanical, Industrial, \& Manufacturing Engineering\\
%Oregon State University\\
%Corvallis, Oregon 97331-6011\\
%Email: bbatten@engr.orst.edu}

%\maketitle
%\begin{abstract}  This course is the second in the ME graduate control sequence on linear methods. The course focuses on developing methods from optimization to derive optimal %controls. Those approaches will be used to develop robust controls such as MinMax and $H_{\infty}$. Some numerical methods will be demonstrated via Matlab, and each student %will be expected to have a project to which they can apply methods discussed in the course.
%\end{abstract}

%\section{Introduction}
%\label{sec:introduction}
%General unconstrained optimization problems from calculus typically take the form of
\begin{equation}
\min f(x)\, {\rm or}\, \max f(x).
\end{equation}
Constraints can be added and the problems take the form of 
$$
\min f(x)\, {\rm or}\, \max f(x).
$$ 
subject to
\begin{equation}
g(x) = C.
\end{equation}
ME 517 considers cases where these problems are algebraic (i.e., no differential equations). The functions $f, g$ can be defined on scalars $x\in \Re$, or vectors $x \in \Re^n$; similar, $f, g$ can take on values in $\Re$ or in $\Re^n$.

In this course, there may be derivatives in either the objective function $f$, or in the constraints $g$. We will develop techniques to address these kinds of problems, and we will need the basics of the Calculus of Variations. The types of problems addressed in this course will be those for which a mathematical model can be developed. 

Examples of optimal control problems:
\begin{enumerate}
\item About 1910, R. H. Goddard asked if we can do space research with rockets. Was involved in the liquid propellant rocket engine whose thrust could be controlled. Question: How should thrust vary with time in order to reach the highest altitude?  \\
Objective function:  maximize altitude at final time, $h(t_f)$\\
Constraint:  with no drag, the thrust should be maximum until no propellant is left.  However, if drag depends on velocity and height, maybe high velocity is not good in densest part of atmosphere. Optimal control can be used to develop a velocity profile.
$v$: velocity, $u$: thrust, $h$: height, $m$: mass, $g$: gravity, $D$: drag
$$\max h(t_f)$$
\hspace{1.5in}{\it subject to}
\begin{eqnarray*}
\dot{v}&=&\frac 1m(u - D(v,h) - g)\\
\dot{m}&=&-\gamma u
\end{eqnarray*}
$$
v(0) = 0, \quad h(0) = 0, \quad m(0) = m_0, \quad m(t_f) = m_1$$
$$0\leq u\leq u_{max}$$

\item Elevated train $m$ on a smooth track with friction constant $\gamma$. Bring train to rest in minimum time with minimum jerk.\\
$x(t)$: position of train; $u(t)$: external force controlling train.
$$\min \sigma t_f + (1-\sigma)\dddot x(t)$$
\hspace{1.5in}{\it subject to}
$$ m\ddot x (t) + \gamma \dot x(t) = u(t)$$
$$x(0) = x_0, \quad \dot x (0) = x_1, \quad x(t_f) = x_d, \quad \dot x(t_f) = 0, \quad  |u(t)| \leq u_{max}$$

\item Linear Quadratic Regulator---infinite time horizon version
$$\min \int_0^{\infty} \left(x^T(t)Qx(t) + u^T(t)Ru(t) \right)dt$$
\hspace{1.5in}{\it subject to}
$$\dot x(t) = Ax(t) + Bu(t), \quad x(0) = x_0$$
$$ x(t) \in \Re^n, \quad u(t) \in \Re^m, \quad A\in\Re ^{n\times n}, \quad B\in \Re ^{n\times m}$$

Example:  $m\ddot y + d\dot y + k y = u(t), \quad y(0) = y_0, \dot y(0) = y_1$
\begin{eqnarray*}
 x_1 = y \qquad \dot x_1 = \dot y &=& x_2 \\
 x_2 =\dot y \qquad \dot x_2 = \ddot y &=& -\frac d m \dot y - \frac km y + \frac 1m u\\
 &=& -\frac d m x_2 - \frac km x_1 + \frac 1m u
\end{eqnarray*}

\begin{eqnarray*}
\dot x = \left[\begin{array}{c} \dot x_1\\ \dot x_2\end{array}\right] = \left[\begin{array}{cc}0  & 1\\  - \frac km &-\frac d m\end{array}\right] \left [ \begin{array}{c} x_1\\ x_2\end{array}\right]  + \left[\begin{array}c 0 \\ \frac 1m\end{array}\right] u
\end{eqnarray*}
 If $\ds Q = \left[\begin{array}{cc} 0&0\\0&1\end{array}\right]$, then velocity is weighted;  if $\ds Q = \left[\begin{array}{cc} 1&0\\0&0\end{array}\right]$, then position is weighted.
 
 



\end{enumerate}

%\section{Linear Quadratic Regulator, take I}
%\label{sec:lqr}
%The Linear Quadratic Regulator (LQR) Problem can be written as
$$\min J(x,u) = \int_0^{T} \left(x^T(t)Qx(t) + u^T(t)Ru(t) \right)dt$$
\hspace{1.5in}{\it subject to}
$$\dot x(t) = Ax(t) + Bu(t), \quad x(0) = x_0$$
$$ x(t) \in \Re^n, \quad u(t) \in \Re^m, \quad A\in\Re ^{n\times n}, \quad B\in \Re ^{n\times m}$$
$$Q\in\Re ^{n\times n}, \quad R\in \Re ^{m\times m}\quad T \hspace{.1in}{\rm fixed}.$$
The matrix $Q$ must be positive semidefinite, and the matrix $R$ must be positive definite for the problem to have a solution.

Since $x$ is a function of $u$, we can consider the objective function alternatively as a function of $u$, i.e., $J(u)$. Using tools which we will develop later in the
course, we can write the solution for this problem in feedback form as 
$$u = Kx$$
where $K = -R^{-1}B^T \Pi$ and $\Pi$ is the solution of the differential Riccati equation
\begin{equation}\label{eq:dre}
\dot{\Pi}(t) +A^T\Pi(t) +\Pi(t)A - \Pi(t)BR^{-1}B^T\Pi(t) +Q =0, \hspace{.1in}  \Pi(T) = 0.
\end{equation}
If we let $T\rightarrow\infty$, the steady state form of the problem is the same, but $\Pi$ is then the solution of the algebraic Riccati equation
\begin{equation}\label{eq:are}
A^T\Pi(t) + \Pi(t)A - \Pi(t)BR^{-1}B^T\Pi(t) +Q=0.
\end{equation}
Note that if $Q$ is positive definite, $\Pi$ is also positive definite.


%
\setcounter{page}{7}
\setcounter{section}{4}
%\section{Optimization Review}
%\label{sec:opt}
%\subsection{Unconstrained Optimization Problems}
\begin{itemize}
\item Functions of a single variable. Let the function $f: \Re \rightarrow \Re$. First order necessary conditions for a minimum at $x = x^*$
$$f'(x) = \frac{d}{dx}f(x^*) = 0.$$
Second order sufficiency conditions
$$f''(x) = \frac{d^2}{dx^2}f(x^*) > 0.$$
If $f:[a,b]\rightarrow\Re$, then the minimum could occur at $a$ or $b$.


\item Functions of multiple variables. Let the function $f: \Re^n \rightarrow \Re$, example $f(x_1, x_2, x_3):\Re^3 \rightarrow \Re$. First order necessary conditions for a minimum at $x = x^*$
$$\grad f(x^*) = 0.$$ Second order sufficiency conditions  $$\grad^2 f(x^*) >0, $$ that is, the matrix is positive definite. To be precise,
$$\grad f(x^*) = \left[ \frac{\pd f}{\pd x_1}(x^*), \frac{\pd f}{\pd x_2}(x^*), ..., \frac{\pd f}{\pd x_n}(x^*)\right]^T = 0$$
$$\grad^2 f(x^*) = \left[\begin{array}{cccc} \frac{\pd^2 f}{\pd x_1^2}(x^*)& \frac{\pd f^2}{\pd x_1 \pd x_2}(x^*)& ... & \frac{\pd f^2}{\pd x_1\pd x_n}(x^*)\\[1em]
\frac{\pd^2 f}{\pd x_2\pd x_1}(x^*)& \frac{\pd f^2}{\pd x_2^2}(x^*)& ... & \frac{\pd f^2}{\pd x_2\pd x_n}(x^*)\\[2em]
\frac{\pd^2 f}{\pd x_n\pd x_1}(x^*)& \frac{\pd f^2}{\pd x_n \pd x_2}(x^*)& ... & \frac{\pd f^2}{\pd x_n^2}(x^*)
\end{array}\right] {\rm positive\, definite}.$$


\end{itemize}
Example problems
\newpage

\subsection{Optimization with Static Equality Constraints}
Standard problem takes the form
$$\min f(x), f:\Re^n \rightarrow \Re$$
\hspace{.15in}{\it subject to}
$$ g(x) = C, g:\Re^n\rightarrow\Re^m$$
Example:  $$\ds \min f(x) = \frac 12 \left(\frac{x_1^2}{a^2}+\frac{x_2^2}{b^2}\right)$$
\hspace{1.5in}{\it subject to} $$x_1 + mx_2 = C$$

\begin{itemize}
\item Solution approach 1: eliminate constraint through substitution and treat as unconstrained problem
\newpage
\item Solution approach 2: look at geometry of problem.  A minimizer can only occur where constrain set is tangent to $f$. At these tangent points, the gradient of $f$ is parallel to the gradient of $g$. Hence, there exists a constant $\lambda$ so that $\grad f(x^*) = \lambda \grad g(x^*)$ (if $g$ maps to $\Re^m$, then $\lambda$ is a vector of size $m$). Define the Lagrangian to be $$H(x) = f(x) + \lambda^Tg(x), \quad {\rm for} \lambda \in \Re^m.$$  The gradient of $H$ is zero at $x^*$ means that the gradient of $f$ is parallel to the gradient of $g$ at $x^*$. 
\end{itemize}
Example continued...
\newpage

%
\section{Optimal Control Problems}
\label{sec:optcont}
\input{optcontrol}
%
%
%\section{Numerical Results}
%\label{sec:results}
%\input{results}
%
%\newpage
%\pagestyle{empty}
%\section*{Notation}
%\label{sec:notation}
%

\begin{itemize}
\item $\Re$:  the set of real numbers, also written as $x\in\Re$ iff $-\infty<x<\infty$
\item $\Re^n = \Re \times \Re \times \cdot \times\Re$; a vector $x \in \Re^n$ iff each element of the vector $x_i\in \Re$.
\item $A\in \Re^{n\times m}$ denotes a matrix with $n$ rows and $m$ columns
\item $C[a,b]$ denotes the set of functions that are continuous on the interval $a\leq x\leq b$
\item $C^n[a,b]$ denotes the set of functions that has $n$ continuous derivatives on the interval $a\leq x \leq b$
\item Leibnitz's formula: $$\ds \frac{d}{dx}\int_{g(x)}^{h(x)}L(x,f(x))dx = \int_{g(x)}^{h(x)}\frac{\pd}{\pd x}L(x,f(x))dx$$ 
 $$ \qquad+ \frac{d h(x)}{d x}L(h(x),f(h(x)))-\frac{d g(x)}{d x}L(g(x),f(g(x))$$

\end{itemize}

%
%----------------------------------------------------------------

%\section*{Acknowledgments}
%This research is supported in part by the Air Force Office of Scientific Research through grants FA9550-05-1-0041 and FA9550-07-1-0540.

\bibliographystyle{aiaa}
\bibliography{AIAA09}


\end{document}


\end{document}

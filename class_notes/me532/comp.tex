In many, if not most real world scenarios, it is impossible to know the entire state of the system, and instead, limited measurements are known.  If we have a system
$$
\dot x(t) = f(t,x(t),u(t)), \quad x(0) = x_0
$$
we denote the sensed measurements (a.k.a. measured output) as
$$y = Cx.$$
Note that $C=I$ implies full state feedback.

Since we know the system dynamics, we can build an estimator (a.k.a observer or filter) to estimate the state, $x_c$ based on the sensed measurements, $y$. This estimator has the form
$$
\dot x_c(t) = A_cx_c(t) + Fy(t) , \quad x_c(0) = x_{c_0}.
$$
The control law has the same feedback gain, but now, instead of feeding back the entire state $x$ (which we do not have), we feed back $x_c$. So, the control law looks like
$$u(t) = -Kx_c(t).$$

These four equations are coupled; by substituting $u$ into the state equation and $y$ into the estimator equation, we obtain the system
$$\left[\begin{array}{c} \dot x\\ \dot x_c\end{array}\right] = \left[\begin{array}{cc}A & -BK \\ FC &A_c\end{array}\right]    \left[\begin{array}{c}  x\\  x_c\end{array}\right] $$

To determine $A_c$, we can use either the approach of Linear Quadratic Gaussian, or MinMax.  The specific optimization problems from which these result will be discussed in a future lesson. At this point, we will simply formally present the equations that come out of the first order necessary conditions for these problems.

Note,  there are several variations on these necessary conditions. We will start with the simplest one and build out from there.

\subsection{Linear Quadratic Gaussian}
This compensator is the extension of LQR applied to problems with limited system information. Assume we have linear system of the form
$$
\dot x(t) = Ax(t) + Bu(t), \quad x(0) = x_0
$$
$$ y(t) = Cx(t).$$
 In this case, the feedback gain matrix $K$, is the same as for LQR:
$$K = -R^{-1} B^T \Pi$$
where $R$ is a symmetric positive definite weighting matrix for the control, and $\Pi$ is the solution to the algebraic Riccati equation
$$A^T\Pi + \Pi A - \Pi BR^{-1}B^T \Pi + Q = 0.$$
Here, $Q$ is a symmetric positive semidefinite weighting matrix for the state.
To determine $F$ and $A_c$, a second Riccati equation (sometimes called the filter Riccati equation) is solved:
$$P A^T + AP + P C^TNC P + BB^T = 0$$
where $N$ is another positive definite weighting matrix.

In this case, $$F = PC^T$$ and $$A_c = A - BK - FC.$$

\subsection{MinMax Compensator}
This compensator is the extension of the MinMax controller for full-state feedback. Assume we have a linear system of the form
$$
\dot x(t) = Ax(t) + Bu(t) + Dw(t), \quad x(0) = x_0
$$
$$ y(t) = Cx(t).$$
 In this case, the feedback gain matrix $K$, is the same as for MinMax:
$$K = -R^{-1} B^T \Pi$$
where $R$ is a symmetric positive definite weighting matrix for the control, and $\Pi$ is the solution to the algebraic Riccati equation
$$A^T\Pi + \Pi A - \Pi (BR^{-1}B^T - \theta^2DD^T) \Pi + Q = 0.$$
Here, $Q$ is a symmetric positive semidefinite weighting matrix for the state, and $\theta$ is the MinMax parameter ($\theta = 0$ yields LQR/LQG).
To determine $F$ and $A_c$, a second Riccati equation is solved:
$$P A^T + AP + P (C^TNC - \theta^2Q)P + DD^T = 0$$
where $N$ is another positive definite weighting matrix.

In this case, $$F = (I - \theta^2 P\Pi)PC^T$$ and $$A_c = A - BK - FC + \theta^2DD^T\Pi.$$
